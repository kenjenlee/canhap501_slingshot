\hypertarget{project-iteration-3}{%
\section{Project Iteration 3}\label{project-iteration-3}}

\hypertarget{goal}{%
\subsection{Goal}\label{goal}}

Add Asteroids that interact with the spaceship.

For the third iteration, our goal was to put the final touches to our
haptically enabled game.

Ken was tasked to add a hotkey to turn the haptics on and off and along
with Anchit they decided to add arrows and text to show the acting
forces on the ship and what keys to press for slingshot and thrustor
mode.

Anchit was tasked add extra haptic feedback, and a tutorial scene. Punit
decided to add an HUD that included a Timer, Fuel Remaining etc.

Finally, I was tasked with fixing the simulation step for rendering the
gravitational forces on the haply, adding the logic for a crossfade
slider between the force of the thrustors and gravitational forces being
felt through the Haply. Finally I also decided to add asteroids in to
give the player something to dodge. Each time an asteroid hits the ship,
some fuel is lost.

\hypertarget{achievements}{%
\subsection{Achievements}\label{achievements}}

To view others results view included iteration posts in appendix.

\begin{itemize}
\tightlist
\item
  Simulation step for rendering graviational forces which means

  \begin{itemize}
  \tightlist
  \item
    creating a selector to audition the gravity of each planet using the
    left and right arrow keys when in free movement mode and
  \item
    adding rendering the forces of the planets on the spaceship when in
    released mode, the mode that comes after being sent flying from the
    slingshot.
  \item
    Finally I added a crossfader equation between the force of the
    planets gravity and the thrusters when in released mode.
  \end{itemize}
\item
  Addition of asteroids:

  \begin{itemize}
  \tightlist
  \item
    Added the code to randomly generate asteroids based on a prefab
    asteroid and some scripts see code for more information.
  \item
    Added collision layers so that the space ship in the slingshot
    scenario does not come in collision with the asteroids until on
    after it is realeased.
  \item
    Made it so that they are destroyed if they hit a planet.
  \item
    Adjusted size and texture of the 'roids
  \item
    Added an asteroid sprite to them.
  \end{itemize}
\end{itemize}

\hypertarget{discussion}{%
\subsection{Discussion}\label{discussion}}

Unfortunatey, most of my work was done away from my hapley, and Montreal
was hit with a horrible freezing rain storm that cripled all the island
execpt for downtown as the powerlines are underground instead of being
on telephone poles. So fortunately I was able to go to my lab to do most
of the work. On that point, fixing the gravity was crucial in having the
game perform as we would more of less want since the goal was to
simulate gravity. Since the generation of gravity uses loops to iterate
through the planets to add together their accelerations, it was not
suited to have that be done in the simulation steps of the haply and
instead be done in it's own process. As such, to add the calculated
gravitational forces into the end-effectors net force variable, I simply
needed to store the gavity value in a discoverable spot to be accessed
during the Haplys' simulation step. I would comment that it is not the
best implementation in terms of security.

When it comes to the asteroids, they add a bit of a bullet dodge
mechanic to the game and having the player loose fuel for every
collision will give them a sense of urgency to reach the end. This of
course is only implemented for the released mode. When the player is in
free movement, they would most likely want to simply observe how the
planets move and interact with everything. Therefore, turning off
collisions with the asteroids is probably best in this situation. Now,
in order to avoid the asteroids from colliding with the ship when it is
stationary in the slingshot, I added collision layers where when static,
the ship is on a layer that does not interact with the asteroids but
when in release mode switches to the layer that allows for interactions.

Another challenge was with how the planets interacted with the
asteroids. Since the planets physics are not being handeled by a
Rigidbody object and instead being directly computed within the script,
the asteroids were passing over them without collisions. So a simple
solution was to a \textbf{sperical collider} to the planets gameObject
to serve as a collision trigger. Like that, when the asteroids collide
with the planets, they are destroyed.

\hypertarget{future-steps}{%
\subsection{Future Steps}\label{future-steps}}

Now that the last additions are being made, the only things I believe
that are needed to be done is the writing of the final report and tuning
of the games variables like the gravitational force as it is always
being modified.
